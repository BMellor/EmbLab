

\documentclass[11pt,fleqn]{book} % Default font size and left-justified equations

\input{structure} % Insert the commands.tex file which contains the majority of the structure behind the template
\usepackage{float}

\usepackage{listings} 
\lstset
{ 
    language=C,
    basicstyle=\ttfamily,
    columns=fullflexible,
    keepspaces=true,
    numbers=none,
    stepnumber=1,
    showstringspaces=false,
    tabsize=1,
    breaklines=true,
    breakatwhitespace=false,
    keywordstyle=\color{blue!80!black},
    stringstyle=\color{red!80!black},
    commentstyle=\color{green!40!black},
    morecomment=[l][\color{magenta!80!black}]{\#}
}

\usepackage{caption}
\captionsetup[figure]{font=small,skip=10pt}

%\usepackage{enumitem}
%\setlist{noitemsep} % or \setlist{noitemsep} to leave space around whole list


%%%%% May be too harsh to prevent paragraph breaks across pages! 
%\interlinepenalty 10000
\widowpenalties 1 10000
\raggedbottom


\newcommand{\ilcode}[1]{
    %\vspace{0.5pt}
    \smallskip
    \colorbox{gray!20!white}{
        \centering
        \parbox{\linewidth-2\fboxsep}{
            \lstinline@#1@
        }
    }
    %\vspace{0.5pt}
}

\newcommand{\code}[3]{
    \begin{figure}[]
        \colorbox{gray!20!white}{
            \parbox{\linewidth-2\fboxsep} {
                \centering 
                \lstinputlisting[language=C]{#1}
            }
        }
        \caption{#2}
        \label{#3}
    \end{figure}
}

\usepackage{textcomp}
\usepackage{wrapfig}
\usepackage{float}

\usepackage{silence} % http://ctan.org/pkg/silence
\ErrorFilter{textcomp}{Symbol \textrightarrow not provided}

% Disable paragraph indentation globally since template was indenting some and not others. (looked terrible)
\setlength{\parindent}{0pt}


%%%%%%%%%%%%%%%%%%%%%%%%%%%%%%%%%%%%%%%%%%%%%%%%%%%%%%%%%%%%%%%%%%%%%%%%%%%%%%%%%%%%%%%%%%%%%%%%%
%%%%                                                                                         %%%%
%%%%       Chapter 6:                                                                        %%%%
%%%%                                                                                         %%%%
%%%%%%%%%%%%%%%%%%%%%%%%%%%%%%%%%%%%%%%%%%%%%%%%%%%%%%%%%%%%%%%%%%%%%%%%%%%%%%%%%%%%%%%%%%%%%%%%%

\setcounter{chapter}{5} % Manually adjust chapter counter to number before desired chapter heading

\begin{document}
	
\chapterimage{chapter_head_2.png} % Chapter heading image
\chapter{The Inter-Integrated Circuit (I\textsuperscript{2}C) Interface}

%\begin{warning}
%  \textbf{NOTE: This lab manual is incomplete!}\\
%  This lab material will be heavily updated, expanded and revised over the next few days. Currently, the material involves communicating with the LM75 temperature sensor that will be used on the motor driver boards. This will be removed and updated to use the L3GD20 MEMS Gyroscope instead. 
%\end{warning}

\section{Introduction to I\textsuperscript{2}C}
\subsection{History of I2C}
I\textsuperscript{2}C (Inter-Integrated Circuit), is a serial communications bus developed by Philips Semiconductor in 1982. It is usually used as a lower-speed interface to connect multiple sensors to a processor. 

\subsection{Operating Modes}
\subsubsection{Master-Slave Architecture}
\subsubsection{Speed Variants}

\section{Electrical Characteristics}
\subsection{Connections and Topology}
\subsection{Push-Pull and Open-Drain Outputs}


\subsubsection{Why Open-Drain for I\textsuperscript{2}C}
Unlike USART connections, the I2C bus drivers are “open drain”, meaning that they can pull the corresponding signal line low, but cannot drive it high. Thus, there can be no bus contention where one device is trying to drive the line high while another tries to pull it low, eliminating the potential for damage to the drivers or excessive power dissipation in the system. Each signal line has a pull-up resistor on it, to restore the signal to high when no device is asserting it low.


\section{Structure of an I\textsuperscript{2}C Transaction}
\begin{figure}[]
    \centering\includegraphics[width=\textwidth]{I2C_Overview}
    \caption{Example I\textsuperscript{2}C transaction.}
    \label{overview}
\end{figure}

\subsection{Overview}
Communication via I2C is more complex than with a USART. The signaling must adhere to a strict protocol for the devices on the bus to recognize it as valid I2C communications.  

Messages are broken up into two types of frame: an address frame, where the master indicates the slave to which the message is being sent, and one or more data frames, which are 8-bit data messages passed from master to slave or vice versa. Data is placed on the data (SDA) line while the clock (SCL) is pulled low, and is sampled during the clock's rising edge. The time between clock edge and data read/write is determined by the devices on the bus and the configuration of the I2C peripheral.

\subsection{Address Frame}
An address frame always begins every I2C transaction. It's purpose is to select a specific slave device to communicate while indicating whether the master wishes to read or write data. 
\subsubsection{Start Condition}
To initiate the address frame, the master device leaves SCL high and pulls SDA low. This notifies all slave devices that a transmission is about to start. If two master devices wish to take ownership of the bus at one time, whichever device pulls SDA low first wins the race and gains control of the bus. 
\subsubsection{Addressing and the Read/Write Bit}
For a 7-bit address, the address is clocked out most significant bit (MSB) first, followed by a read/write bit indicating whether this is a read (1) or write (0) operation. 
\subsubsection{Slave Acknowledgment}
The 9th bit of the frame is the NACK/ACK bit. This is the case for all frames (data or address). Once the first 8 bits of the frame are sent, the receiving device is given control over SDA. If the receiving device does not pull the SDA line low before the 9th clock pulse, it can be inferred that the receiving device either did not receive the data or did not know how to parse the message. In that case, the exchange halts, and it’s up to the master of the system to decide how to proceed.

\subsection{Data Frame}
After the address frame has been sent, data can begin being transmitted. The master will simply continue generating clock pulses at a regular interval, and the data will be placed on SDA by either the master or the slave, depending on whether the R/W bit indicated a read or write operation. The number of data frames is arbitrary, and most slave devices will auto-increment the internal register, meaning that subsequent reads or writes will come from the next register in line.
\subsubsection{Stop Condition}
Once all the data frames have been sent, the master will generate a stop condition. Stop conditions are defined by a 0->1 (low to high) transition on SDA after a 0->1 transition on SCL, with SCL remaining high. During normal data writing operation, the value on SDA should not change when SCL is high, to avoid false stop conditions.

\subsection{Multi-Transaction Communications}
 By design, I2C doesn't allow reading and writing data within a single transaction. However, it is possible for a master device to chain multiple transactions together without releasing the bus. This has the benefit that no other devices can claim the bus during the preparation for the next transaction. 
\subsubsection{Restart Conditions}
To chain multiple transactions together, the master generates a restart condition.
 To perform a restart condition, SDA is allowed to go high while SCL is low, SCL is allowed to go high, and then SDA is brought low again while SCL is high. Because there was no stop condition on the bus, the previous communication wasn’t truly completed and the current master maintains control of the bus.
 
 At this point, the next message can begin transmission. The syntax of this new message is the same as any other message–an address frame followed by data frames. Any number of repeated starts is allowed, and the master will maintain control of the bus until it issues a stop condition.
\begin{figure}[]
    \centering\includegraphics[width=\textwidth]{I2C_Restart}
    \caption{Repeated start conditions allow for chaining multiple transactions. }
    \label{restart}
\end{figure}

\section{Using the L3GD20 MEMS Gyroscope}

\section{Introducing the I\textsuperscript{2}C Peripheral}
\subsection{I\textsuperscript{2}C Registers}
\begin{itemize}
    \item \textbf{Control register 1 (I2C\_CR1)} -- The control register which manages I2C system-wide configuration settings. You'll be configuring most of the main parameters here.
    \item \textbf{Control register 2 (I2C\_CR2)} -- The control register which sets parameters for current I2C transaction. You'll be configuring your reads/writes here.
    \item \textbf{Timing register (I2C\_TIMINGR)} -- This register controls all of the I2C timing parameters, the values written here need to either be derived from a system of equations manually or from the example timings table located in the datasheet. 
    \item \textbf{Interrupt and status register (I2C\_ISR)} -- The interrupt and status register, this register is read-only.
    \item \textbf{Interrupt clear register (I2C\_ICR)} -- This register can be used to clear the flags in the ISR.
    \item \textbf{Receive data register (I2C\_RXDR)} -- The receive data register
    \item \textbf{Transmit data register (I2C\_TXDR)} -- The transmit data register
\end{itemize}

\subsection{Initializing the Peripheral}

\subsection{}

\section{Debugging I\textsuperscript{2}C Communications}

The figure \ref{fullCaptureAnnotated} shows a logic capture of a successful reading of the chip ID register, we'll go over important parts of the image afterwards. 

\begin{figure}[]
    \centering\includegraphics[width=\textwidth]{fullCaptureAnnotated}
    \caption{Annotated full I2C capture}
    \label{fullCaptureAnnotated}
\end{figure}

The first three things to notice in the above figure is that the I2C protocol analyzer marks every START (green dot) and STOP (brown dot) condition. The initial START (circled in red) sets up a write to the LM75 to select the chip ID register. Then without releasing the bus (demonstrated a bit later) the peripheral issues a second START/RESTART (circled in yellow) to begin the data read. Finally the peripheral issues a STOP (circled in orange) condition to release the bus.  

One thing to notice is that after every data packet the analyzer lists whether it received an ACK or NACK. The I2C peripheral will set flags in its status register to let you know whether something responded to a packet or not. The final packet in a I2C transaction typically isn't acknowledged. Figure \ref{noRespCapture} shows a capture where the LM75 wasn't connected. If you see this in your debugging either you have a wrong slave address, the slave isn't powered or connected properly or you have the I2C pins not configured to open-drain output mode. 

\begin{figure}[]
    \centering\includegraphics[width=\textwidth]{noRespCapture}
    \caption{I2C capture without slave response}
    \label{noRespCapture}
\end{figure}

One common mistake is to forget to release the I2C clock line (SCL) with a STOP condition after a transaction. This will prevent any further communications from being able to use the bus. The idle state of the I2C lines should always be high. Figure \ref{noStopCapture} shows a trace where the code is missing the final STOP condition. (the portion circled in red should have returned high)


\begin{figure}[]
    \centering\includegraphics[width=\textwidth]{noStopCapture}
    \caption{Final data frame without stop condition}
    \label{noStopCapture}
\end{figure}

Finally, in an I2C RESTART condition such as between a write and a read, the peripheral purposely doesn't issue a STOP because it doesn't want to surrender the bus to anyone else that may be waiting to use it. Typically the space between the two operations (read and write) is small enough that you can't see it except for the long repeated start cycle on the clock. Figure \ref{restartCapture} shows a trace where there is some delay between the two operations. 

\begin{figure}[]
    \centering\includegraphics[width=\textwidth]{restartCapture}
    \caption{Repeated start condition (restart)}
    \label{restartCapture}
\end{figure}

The portion circled in red is the released/idle state of the bus, the orange circled portion is where the peripheral is purposefully holding the bus active until it's ready to communicate again.


\section{Lab Assignment}

\end{document}
